\documentclass{article}%
\usepackage[T1]{fontenc}%
\usepackage[utf8]{inputenc}%
\usepackage{lmodern}%
\usepackage{textcomp}%
\usepackage{lastpage}%
\usepackage{geometry}%
\geometry{head=40pt,margin=5mm,bottom=0.6cm,includeheadfoot=True}%
\usepackage{tkz-tab}%
\usepackage{amsmath}%
\usepackage{ragged2e}%
\usepackage{fancyhdr}%
%
\usepackage{newunicodechar}%
\newunicodechar{∞}{\ensuremath{\infty}}%
\newunicodechar{Δ}{\ensuremath{\Delta}}%
\newunicodechar{α}{\ensuremath{\alpha}}%
\newunicodechar{β}{\ensuremath{\beta}}%
\fancypagestyle{header}{%
\renewcommand{\headrulewidth}{0pt}%
\renewcommand{\footrulewidth}{0pt}%
\fancyhead{%
}%
\fancyfoot{%
}%
\fancyhead[C]{%
\begin{Huge}%
\textbf{Equation Inéquation premier degre}%
\end{Huge}%
}%
\fancyfoot[L]{%
\textit{Théo LUBAN}%
}%
\fancyfoot[R]{%
\textit{Quentin PLADEAU}%
}%
}%
%
\begin{document}%
\normalsize%
\pagestyle{header}%
\begin{minipage}{\textwidth}%
\centering%
\vspace*{200pt}%
\fontsize{50}{36}%
\selectfont%
\textbf{Python{-}Maths}%
\linebreak%
\fontsize{30}{24}%
\selectfont%
Générateur d'exercices avec leurs corrections%
\end{minipage}%
\newpage%
\fontsize{12}{10}%
\selectfont%
\section*{Equation du premier degré n°1}%
\label{sec:Equationdupremierdegrn1}%
Avec les équations du premier degré suivantes : %
\subsection*{Equation niveau 1 : }%
\label{subsec:Equationniveau1}%
\ $38 - 26x = 24 + 30 $%
\vspace{5mm}%
\newline%
\ $13 + 33x  = 34 + 49$%
\vspace{5mm}%
\newline%
\ $45 + 3 = 18x - 31$

%
\subsection*{Equation niveau 2 : }%
\label{subsec:Equationniveau2}%
$\frac{38}{26}x + 24 = \frac{30}{13} - 33x$%
\vspace{5mm}%
\newline%
$\ 34 + \frac{49}{45}x = \frac{3}{18}$%
\vspace{5mm}%
\newline%
$\frac{38}{24}x - 13 = \frac{34}{45} - 31x$%
\vspace{5mm}%
\newline

%
\newpage%
\section*{Correction Equation du premier degré n°1}%
\label{sec:CorrectionEquationdupremierdegrn1}%
Niveau 1 :%
\subsection{bon jour}%
\label{subsec:bonjour}%
Equations n°1%
\begin{align*}%
\vspace{10mm}%
\newline%
\  \Leftrightarrow 38 - 26x &= 24 + 30     & \Leftrightarrow 13 + 33x  &= 34 + 49       & \Leftrightarrow 45 + 3 &= 18x - 31\\%
\vspace{5mm}%
\newline%
\  \Leftrightarrow 38 - 38 - 26x &= 24 + 30 -38      & \Leftrightarrow 13 - 13 + 33x  &= 34 + 49 + 13        & \Leftrightarrow 45 + 3 + 31 &= 18x - 31 +31\\%
\vspace{5mm}%
\newline%
\  \Leftrightarrow \frac{-26x}{-26} &= \frac{16}{-26}      & \Leftrightarrow \frac{33x}{33} &= \frac{96}{33}      & \Leftrightarrow \frac{79}{18} &= \frac{18x}{18}\\%
\vspace{5mm}%
\newline%
\  \Leftrightarrow  x &= -0.6      & \Leftrightarrow  x &= 2.9      & \Leftrightarrow  x &= 4.4\\%
\end{align*}

%
\newpage%
\end{document}
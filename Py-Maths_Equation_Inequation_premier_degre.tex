\documentclass{article}%
\usepackage[T1]{fontenc}%
\usepackage[utf8]{inputenc}%
\usepackage{lmodern}%
\usepackage{textcomp}%
\usepackage{lastpage}%
\usepackage{geometry}%
\geometry{head=40pt,margin=5mm,bottom=0.6cm,includeheadfoot=True}%
\usepackage{tkz-tab}%
\usepackage{amsmath}%
\usepackage{ragged2e}%
\usepackage{fancyhdr}%
%
\usepackage{newunicodechar}%
\newunicodechar{∞}{\ensuremath{\infty}}%
\newunicodechar{Δ}{\ensuremath{\Delta}}%
\newunicodechar{α}{\ensuremath{\alpha}}%
\newunicodechar{β}{\ensuremath{\beta}}%
\fancypagestyle{header}{%
\renewcommand{\headrulewidth}{0pt}%
\renewcommand{\footrulewidth}{0pt}%
\fancyhead{%
}%
\fancyfoot{%
}%
\fancyhead[C]{%
\begin{Huge}%
\textbf{Equation\_Inequation\_premier\_degre}%
\end{Huge}%
}%
\fancyfoot[L]{%
\textit{Théo LUBAN}%
}%
\fancyfoot[R]{%
\textit{Quentin PLADEAU}%
}%
}%
%
\begin{document}%
\normalsize%
\pagestyle{header}%
\begin{minipage}{\textwidth}%
\centering%
\vspace*{200pt}%
\fontsize{50}{36}%
\selectfont%
\textbf{Python{-}Maths}%
\linebreak%
\fontsize{30}{24}%
\selectfont%
Générateur d'exercices avec leurs corrections%
\end{minipage}%
\newpage%
\fontsize{12}{10}%
\selectfont%
\section*{Equation du premier degré n°1}%
\label{sec:Equationdupremierdegrn1}%
Avec les équations du premier degré suivantes : %
\subsection*{Equation niveau 1 : }%
\label{subsec:Equationniveau1}%
\ $34 - 43x = 31 + 49 $%
\vspace{5mm}%
\newline%
\ $9 + 13x  = 30 + 47$%
\vspace{5mm}%
\newline%
\ $48 + 17 = 34x - 22$

%
\subsection*{Equation niveau 2 : }%
\label{subsec:Equationniveau2}%
$\frac{34}{43}x + 31 = \frac{49}{9} - 13x$%
\vspace{5mm}%
\newline%
$\ 30 + \frac{47}{48}x = \frac{17}{34}$%
\vspace{5mm}%
\newline%
$\frac{34}{31}x - 9 = \frac{30}{48} - 22x$%
\vspace{5mm}%
\newline

%
\newpage%
\section*{Correction Equation du premier degré n°1}%
\label{sec:CorrectionEquationdupremierdegrn1}%
Niveau 1 :%
\subsection{Equations n°1}%
\label{subsec:Equationsn1}%
\begin{align*}%
\  \Leftrightarrow 34 - 43x &= 31 + 49     & \Leftrightarrow 9 + 13x  &= 30 + 47       & \Leftrightarrow 48 + 17 &= 34x - 22\\%
\vspace{5mm}%
\newline%
\  \Leftrightarrow 34 - 34 - 43x &= 31 + 49 -34      & \Leftrightarrow 9 - 9 + 13x  &= 30 + 47 + 9        & \Leftrightarrow 48 + 17 + 22 &= 34x - 22 +22\\%
\vspace{5mm}%
\newline%
\  \Leftrightarrow \frac{-43x}{-43} &= \frac{46}{-43}      & \Leftrightarrow \frac{13x}{13} &= \frac{86}{13}      & \Leftrightarrow \frac{87}{34} &= \frac{34x}{34}\\%
\vspace{5mm}%
\newline%
\  \Leftrightarrow  x &= -1.1      & \Leftrightarrow  x &= 6.6      & \Leftrightarrow  x &= 2.6\\%
\end{align*}

%
\subsection{Equations n°2}%
\label{subsec:Equationsn2}%
\begin{align*}%
\  \Leftrightarrow \frac{34}{43}x + 31 &= \frac{49}{9} - 13x       &     \Leftrightarrow 30 + \frac{47}{48}x &= \frac{17}{34} \\%
\vspace{5mm}%
\newline%
\  \Leftrightarrow \frac{34}{43}x + 31 + 13x  &= \frac{49}{9} - 13x +13x    &    \Leftrightarrow 30 - 30 + \frac{47}{48}x &= \frac{17}{34} + 30\\%
\vspace{5mm}%
\newline%
\  \Leftrightarrow 13.8x + 31 - 31 &= \frac{49}{9}	- 31       &      \Leftrightarrow \frac{47}{48}x &= 30.5 \\%
\vspace{5mm}%
\newline%
\  \Leftrightarrow \frac{13.8x}{13.8} &= \frac{-25.6}{13.8}     &      \Leftrightarrow \frac{47}{48}x \times \frac{48}{47} &= 30.5 x \times \frac{48}{47}\\%
\vspace{5mm}%
\newline%
\  \Leftrightarrow x &= -1.9    &     \Leftrightarrow x &= 31.1 \\%
\vspace{5mm}%
\newline%
\end{align*}

%
\newpage%
\end{document}
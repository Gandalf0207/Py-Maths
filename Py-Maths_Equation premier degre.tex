\documentclass{article}%
\usepackage[T1]{fontenc}%
\usepackage[utf8]{inputenc}%
\usepackage{lmodern}%
\usepackage{textcomp}%
\usepackage{lastpage}%
\usepackage{geometry}%
\geometry{head=40pt,margin=5mm,bottom=0.6cm,includeheadfoot=True}%
\usepackage{tkz-tab}%
\usepackage{amsmath}%
\usepackage{ragged2e}%
\usepackage{fancyhdr}%
%
\usepackage{newunicodechar}%
\newunicodechar{∞}{\ensuremath{\infty}}%
\newunicodechar{Δ}{\ensuremath{\Delta}}%
\newunicodechar{α}{\ensuremath{\alpha}}%
\newunicodechar{β}{\ensuremath{\beta}}%
\fancypagestyle{header}{%
\renewcommand{\headrulewidth}{0pt}%
\renewcommand{\footrulewidth}{0pt}%
\fancyhead{%
}%
\fancyfoot{%
}%
\fancyhead[C]{%
\begin{Huge}%
\textbf{Equation premier degre}%
\end{Huge}%
}%
\fancyfoot[L]{%
\textit{Théo LUBAN}%
}%
\fancyfoot[R]{%
\textit{Quentin PLADEAU}%
}%
}%
%
\begin{document}%
\normalsize%
\pagestyle{header}%
\begin{minipage}{\textwidth}%
\centering%
\vspace*{200pt}%
\fontsize{50}{36}%
\selectfont%
\textbf{Py{-}Maths}%
\linebreak%
\fontsize{30}{24}%
\selectfont%
Générateur d'exercices avec leurs corrections%
\end{minipage}%
\newpage%
\fontsize{12}{10}%
\selectfont%
\section*{Exo Equation du premier degré n°1}%
\label{sec:ExoEquationdupremierdegrn1}%
Avec les équations du premier degré suivantes : %
\subsection*{Equation niveau 1 : }%
\label{subsec:Equationniveau1}%
\ $18 - 2x = 50 + 20 $%
\vspace{5mm}%
\newline%
\ $38 + 16x  = 2 + 38$%
\vspace{5mm}%
\newline%
\ $24 + 20 = 9x - 29$

%
\subsection*{Equation niveau 2 : }%
\label{subsec:Equationniveau2}%
$\frac{18}{2}x + 50 = \frac{20}{38} - 16x$%
\vspace{5mm}%
\newline%
$\ 2 + \frac{38}{24}x = \frac{20}{9}$%
\vspace{5mm}%
\newline%
$\frac{18}{50}x - 38 = \frac{2}{24} - 29x$%
\vspace{5mm}%
\newline

%
\subsection*{Equation niveau 3 : }%
\label{subsec:Equationniveau3}%
$\ 18(2x + 50) + 20x = 38(16x - 2)$%
\vspace{5mm}%
\newline%
$\ \frac{-38}{24x + 20} = \frac{9}{29}$%
\vspace{5mm}%
\newline

%
\newpage%
\section*{Correction Exo Equation du premier degré n°1}%
\label{sec:CorrectionExoEquationdupremierdegrn1}%
\subsection*{Equations niveau 1}%
\label{subsec:Equationsniveau1}%
\begin{align*}%
\  \Leftrightarrow 18 - 2x &= 50 + 20     & \Leftrightarrow 38 + 16x  &= 2 + 38       & \Leftrightarrow 24 + 20 &= 9x - 29\\%
\vspace{5mm}%
\newline%
\  \Leftrightarrow 18 - 18 - 2x &= 50 + 20 -18      & \Leftrightarrow 38 - 38 + 16x  &= 2 + 38 + 38        & \Leftrightarrow 24 + 20 + 29 &= 9x - 29 +29\\%
\vspace{5mm}%
\newline%
\  \Leftrightarrow \frac{-2x}{-2} &= \frac{52}{-2}      & \Leftrightarrow \frac{16x}{16} &= \frac{78}{16}      & \Leftrightarrow \frac{73}{9} &= \frac{9x}{9}\\%
\vspace{5mm}%
\newline%
\  \Leftrightarrow  x &= -26.0      & \Leftrightarrow  x &= 4.9      & \Leftrightarrow  x &= 8.1\\%
\end{align*}

%
\subsection*{Equations niveau 2}%
\label{subsec:Equationsniveau2}%
\begin{align*}%
\  \Leftrightarrow \frac{18}{2}x + 50 &= \frac{20}{38} - 16x       &     \Leftrightarrow 2 + \frac{38}{24}x &= \frac{20}{9}     &     \Leftrightarrow  \frac{18}{50}x - 38 &= \frac{2}{24} - 29x\\%
\vspace{5mm}%
\newline%
\  \Leftrightarrow \frac{18}{2}x + 50 + 16x  &= \frac{20}{38} - 16x +16x    &    \Leftrightarrow 2 - 2 + \frac{38}{24}x &= \frac{20}{9} + 2      &     \Leftrightarrow  \frac{18}{50}x +29x - 38 &= \frac{2}{24} - 29x  + 29x\\%
\vspace{5mm}%
\newline%
\  \Leftrightarrow 25.0x + 50 - 50 &= \frac{20}{38}	- 50       &      \Leftrightarrow \frac{38}{24}x &= 4.2     &   \Leftrightarrow  38.0x - 38 +38 &= \frac{2}{24} + 38\\%
\vspace{5mm}%
\newline%
\  \Leftrightarrow \frac{25.0x}{25.0} &= \frac{-49.5}{25.0}     &      \Leftrightarrow \frac{38}{24}x \times \frac{24}{38} &= 4.2 x \times \frac{24}{38}      &      \Leftrightarrow  \frac{38.0x}{38.0} &= \frac{38.1}{38.0}\\%
\vspace{5mm}%
\newline%
\  \Leftrightarrow x &= -2.0    &     \Leftrightarrow x &= 2.7     &     \Leftrightarrow  x &= 1.0\\%
\vspace{5mm}%
\newline%
\end{align*}

%
\subsection*{Equations niveau 3}%
\label{subsec:Equationsniveau3}%
\begin{align*}%
\ 18(2x + 50) + 20x &= 38(16x - 2)  \\%
\vspace{5mm}%
\newline%
\ 18 \times 2x + 18 \times 50 + 20x     &=      38 \times 16x - 38 \times 2 \\%
\vspace{5mm}%
\newline%
\ 56x + 900   &= 608x - 76 \\%
\vspace{5mm}%
\newline%
\ 56x -608x + 900 - 900   &=  608x - 608x + 76 - 900 \\%
\vspace{5mm}%
\newline%
\ \frac{-552x}{-552}   &= \frac{-824}{-552}\\%
\vspace{5mm}%
\newline%
\ x  &= 1.5\\%
\vspace{5mm}%
\newline%
\end{align*}

%
\newpage%
\end{document}